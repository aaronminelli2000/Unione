\section{Conclusioni (messa a confronto e riflessioni personali in risposta alle domande di ricerca)}
In questa parte di progetto, mi sono occupato di approfondire le mie conosce in merito alla comunicazione politica tramite i social media nel Canton Ticino, sia quella istituzionale che privata. La prima riguarda l’amministrazione cantonale, mentre la seconda di due politici che fanno parte del PPD, ovvero il direttore DSS Paolo Beltraminelli e Fabio Bacchetta-Cattori.\\
In seguito al recupero e alla stesura delle informazioni raccolte, sono in grado di rispondere alle mie sotto-domande di ricerca. I social media come strumento comunicativo nel mondo politico sono degli strumenti molto complicati da utilizzare, in quanto se vengono usati nel modo sbagliato possono influenzare negativamente la carriera o addirittura la vita di un politico. Quindi, l’uso di queste piattaforme da parte di persone che hanno un’immagine pubblica, sia a livello locale che mondiale, deve essere svolta con molta cautela. Sarebbe utile un aiuto da parte di una persona esterna, la quale è specializzata nell’uso dei social media, in modo tale da diminuire il rischio di pubblicare dei contenuti falsi o che mettano a rischio la propria reputazione. Però se la comunicazione politica tramite i social media, venisse eseguita in maniera mirata, regolare e soprattutto tenendo presente la quantità di persone iscritte, potrebbe portare del valore aggiunto al politico. Per avere dei riscontri positivi dagli utenti è però necessario pubblicare dei post che abbiano dei contenuti ben precisi e specialmente farlo regolarmente, in modo tale da “fidelizzare” la popolazione che ti segue. Inoltre bisogna differenziare l’utilizzo in ambito politico di questi strumenti in privato e istituzionale. Il primo, risulta un po’ più semplice, in quanto ti permette di mantenere un contatto con i propri follower, dato che essi possono identificarsi in una persona. Mentre per la seconda è più complicato entrare con successo in questo mondo dato che non rappresenta una persona fisica bensì una istituzione. Quindi a differenza di un profilo privato di un politico non può esporsi pubblicamente con la pubblicazione di post provocatori su un determinato tema, mentre il politico può sfruttare dei contenuti un po’ sopra le righe per attirare l’attenzione degli utenti e far parlare di sé. Questa è la principale differenza tra i due tipi di comunicazione, ed è quella più importante perché se una comunicazione di tipo istituzionale adotterebbe una campagna comunicativa troppo aggressiva o provocatoria su un argomento che viene percepito come molto importante dalla popolazione, ad esempio quello del flusso migratorio che sta colpendo l’Europa è uno Governo intero a rovinare la propria immagine pubblica e non una sola come succederebbe se a pubblicare i medesimi contenuti fosse un politico solo, tramite i propri profili privati. In generale, se usati correttamente i social media possono aiutare un politico sia ad aumentare la propria visibilità che a promuoversi in vista di possibili elezioni. Questi strumenti comunicativi sono molto potenti e potenzialmente efficaci, di conseguenza bisogna essere in grado di utilizzarli correttamente. 
