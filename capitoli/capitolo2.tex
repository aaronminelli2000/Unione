\section{Motivazioni, pro e contro l’utilizzo dei social media da parte di Paolo Beltraminelli e Fabio Bacchetta-Cattori}
\subsection{Elenco di quale piattaforme utilizzano e i pro di esse}
Fabio Bacchetta-Cattori nel corso degli anni ha preso in seria considerazione l’utilizzo dei social media, infatti suo figlio gli ha aperto un profilo Facebook che ha utilizzato di rado e non per promuoversi in ambito politico o per comunicare con i propri votanti. Però riconosce la forza di questi mezzi comunicativi e che possono permettere al politico di farsi conoscere oppure ad incrementare il proprio elettorato. Un ulteriore punto a favore di questi strumenti di comunicazione è la velocità con cui gli utenti arrivano alle notizie. Infatti le persone grazie ai propri smartphone sono sempre connesse alla rete e possono quindi reperire le informazioni in tempi molto brevi. Questo punto va a discapito di giornali o telegiornali dato che le persone non devono più aspettare la pubblicazione o la messa in onda per ricevere tutte le notizie che sono successe durante il giorno. Inoltre, secondo lui i social media sono una conseguenza dell’evoluzione tecnologica dato che quando lui era un giovane politico avevano un piccolo spazio chiamato Diario Tazebau sul giornale Eco di Locarno dove potevano esprimere la loro opinione sui temi politici che erano di attualità in quel periodo. Indipendentemente dall’utilizzo dei social media Fabio Bacchetta-Cattori è soddisfatto dal proprio numero di elettori, in quanto il numero è in significativo aumento rispetto anche a chi utilizza questi strumenti comunicativi.
Con Paolo Beltraminelli abbiamo potuto parlare maggiormente sull’utilizzo dei social media in quanto lui li utilizza molto. In particolar modo Facebook, Twitter e Instagram. Ci ha confessato che il motivo principale che lo ha spinto ad utilizzare questi strumenti comunicativi è la curiosità e una volta visti i primi feedback positivi ha continuato ad utilizzarli. In tutto questo tempo ha sempre gestito i propri profili da solo, pensa, fa e pubblica tutti i post da solo, in questo modo riesce a rimanere se stesso anche nei contenuti che pubblica, ovvero riesce a mantenere una parte umana nelle relazioni con gli utenti. Ed essa è una cosa molto importante per riscuotere delle buone reazioni. Uno dei vantaggi dei social media ha potuto constatarlo nel corso di questi anni, ovvero che questi mezzi di comunicazione sono molto più immediati rispetto ai giornali dove si cura molto di più la forma e di conseguenza ci si impiega molto più tempo per scriverli. I post possono essere utilizzati con il fine di creare un dibattito tramite i commenti, in questi casi gli utenti che esprimono la loro opinione sono coloro che si discostano dal messaggio che si vuole dare, dato che solitamente chi si trova d’accordo lascia semplicemente un “mi piace”. Secondo Paolo Beltraminelli sui social media bisogna essere in grado di saper differenziare i contenuti che si pubblicano sui propri profili. Ad esempio su Facebook Paolo Beltraminelli ha due profili, uno è più personale mentre l’altro più istituzionale. Generalmente utilizza il primo in quanto gli permette di mantenere una relazione più diretta con gli utenti che lo seguono. Mentre il secondo viene utilizzato per raccogliere dei dati/delle statistiche sui contenuti che vengono pubblicati. Per esempio quante persone l’hanno visto, quante l’hanno condiviso, ecc. Questa cosa risulta molto interessante dato che permette a Paolo Beltraminelli di verificare se i suoi follower hanno apprezzato o no i contenuti che sono stati pubblicati oppure se gli hanno compresi. In questo modo è in grado di comprendere meglio cosa piace o cosi si aspettano le persone che lo seguono su questo social media. Tramite Facebook sono efficaci sia le foto e i video che la scrittura, quindi è un po' un punto d’incontro tra Instagram e Twitter. Infatti su Instagram sono più efficaci le foto o i video, mentre su Twitter che è frequentato da persone più predisposte a sviluppare i temi che vengono pubblicati è molto più efficace la scrittura anche se la limitazione dei caratteri è maggiore rispetto alle altre piattaforme. 
\subsection{Elenco dei contro di queste piattaforme}
La discussione con Fabio Bacchetta-Cattori si è sviluppata inizialmente sulla motivazione della sua scelta di utilizzare i metodi di comunicazioni più tradizionali quindi in particolare i giornali e non modernizzarsi in modo da comunicare con gli elettori più velocemente come molti suoi colleghi hanno fatto e fanno tuttora. La ragione principale per cui Fabio Bacchetta-Cattori non ha utilizzato i social media è stata che essi sono un mondo virtuale in cui sono presenti solamente cose intangibili e lui vuole mantenere delle relazioni dirette con i propri elettori è ciò non sarebbe possibile tramite questi strumenti di comunicazione. Secondo Fabio Bacchetta-Cattori per sfruttare al meglio i social media in ambito politico, bisognerebbe solamente scambiarsi delle informazioni, ma questo, quando si parla di alcuni temi, come la forte migrazione che sta colpendo l’Europa non viene fatto. Anzi spesso si banalizza su questi argomenti che toccano la dignità delle persone coinvolte e se questi “giochetti” vengono effettuati da grandi politici che sono seguiti da moltissime persone e in questo modo l’obiettivo di questi strumenti non viene centrato. Così facendo i politici sfruttano sia l’argomento che la loro influenza mediatica per accaparrarsi i voti degli elettori.
Per Paolo Beltraminelli una problematica derivante l’uso di queste piattaforme comunicative per un politico è quello di prestare molta attenzione a ciò che viene detto e pubblicato in quanto potrebbe venir frainteso da parte delle persone che leggono i post e questo porterebbe delle problematiche perché si potrebbe creare un’immagine falsa del politico e quindi ne risentirebbe la reputazione e così la sua carriera politica potrebbe venire rovinata. Infatti le discussioni tramite questi mezzi di comunicazioni non possono essere considerate come quelle che vengono fatte tra persone fisiche e non virtuali, anche perché sui social media si può incontrare dei profili falsi che sono stati creati apposta per criticare la gente. Ci sono inoltre altri aspetti che possono rendere i social media degli strumenti negativi e soprattutto pesanti, le fake news ne rappresentano un esempio perfetto. Esse sono molto pericolose in quanto, la diffusione delle notizie sia vere che false, mediante questi mezzi comunicativi avviene molto rapidamente. Ciò può provocare dei seri danni di immagine al politico, in quanto una gran parte della popolazione lo verrà a sapere, ma il vero problema non è alla pubblicazione di queste informazioni finte, bensì in seguito, quando un gran numero di utenti saranno pronti a commentare dei post in maniera negativa e magari anche offensiva solamente perché hai pubblicato un post o un tweet in cui venivano riportate delle notizie inesistenti. Ci sono inoltre delle fake news che durano molto tempo ad esempio il riscaldamento climatico, che il livello di dell’inquinamento dell’aria in Ticino è inferiore ora rispetto a vent’anni fa quando c’era la benzina a piombo però la percezione di questo problema è molto alto, mentre ci sono altre che durano molto meno e che dopo pochi giorni non ricorda più nessuno . Ed è proprio la percezione della gente che ne varia lo sviluppo e la durata di queste notizie. Di conseguenza i politici devono sempre stare molto attenti alle notizie che pubblicano sui propri profili in quanto potrebbero risultare false e poi scatenare tutta una serie di avvenimenti poco piacevoli.  
